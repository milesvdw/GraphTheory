\documentclass[]{article}
\usepackage{graphicx}
\usepackage{amsmath}
\usepackage{amsfonts}
\usepackage{amssymb}
\usepackage{epstopdf}
\usepackage[margin=0.5in]{geometry}
\usepackage[T1]{fontenc}
\newcommand\tab[1][12pt]{\hspace*{#1}}
% Title Page
\title{Homework Set 3}
\author{Miles Van de Wetering, Charles Hill, Cierra Shawe}


\begin{document}
\maketitle

\section*{Problem 1 - a}
How could we delete an arbitrary vertex v from this graph, without changing the shortest-path distance
between any other pair of vertices? Describe an algorithm that constructs a directed graph G' = (V', E') with weighted edges, where V = V $\textbackslash$\{v\}, and the shortest-path distance between any two nodes in G' is equal to the shortest-path distance between the same two nodes in G, in O($V^2$) time.\\

To solve this, we can replace paths that are broken by the removal of \textit{v}. If \textit{v} has no outgoing edges, it is at the end of every path it's a part of (i.e., \textit{v} is a \textit{sink}). Therefore, it can be deleted without affecting any other pair of vertices. \textit{v} has no incoming edges, it is the beginning of every path it's a part of (i.e., \textit{v} is a \textit{source}). Therefore, it can be deleted without affecting any other pair of vertices. If, however, \textit{v} has \textit{at least} one incoming \textit{and} outgoing edge, then we must replace the paths that are impacted by the deletion of \textit{v}. The algorithm for this replacement is as follows: \\

For each edge pred(\textit{v})$\rightarrow$\textit{v}$\rightarrow$succ(\textit{v}), if the successor to \textit{v} is adjacent to the predecessor ov \textit{v}, then record the weight of that edge. If the total weight of edge pred(\textit{v})$\rightarrow$\textit{v}$\rightarrow$succ(\textit{v}) is less than the weight of pred(\textit{v})$\rightarrow$succ(\textit{v}), change the weight of  pred(\textit{v})$\rightarrow$succ(\textit{v}) to weight(pred(\textit{v})$\rightarrow$\textit{v}$\rightarrow$succ(\textit{v})). Then, delete \textit{v}. If, however, succ(\textit{v}) $\notin$ adj(pred(\textit{v)}), add an edge pred(\textit{v})$\rightarrow$succ(\textit{v}) with weight(pred(\textit{v})$\rightarrow$\textit{v}$\rightarrow$succ(\textit{v})). Then, delete \textit{v}. \\

In this way, we preserve the shortest path from each pred(\textit{v}) to each succ(\textit{v}).\\ 

This algorithm runs max 2V times (for each edge incoming and outgoing from \textit{v}), and for each of those 2V runs may have to replace up to 2V edges. Thus, the runtime is O(4$V^2$), or O($V^2$)  \\

\noindent \textbf{Pseudocode:}

\noindent for a random vertex v in V: \\
$\tab$if \textit{v} has no outgoing edges:\\
$\tab$$\tab$ delete \textit{v}\\
$\tab$$\tab$ quit\\
$\tab$if \textit{v} has no incoming edges:\\
$\tab$$\tab$ delete \textit{v}\\
$\tab$$\tab$ quit\\
$\tab$for all edges pred(v)->v->succ(v) \\
$\tab$$\tab$if succ(v) in adj(pred(v)):\\
$\tab$$\tab$$\tab$old.weight = weight pred(v)->succ(v) \\
$\tab$$\tab$if old.weight == null:\\
$\tab$$\tab$$\tab$insert new edge pred(v) -> succ(v)\\
$\tab$$\tab$if old.weight != Null AND weight(pred(v)->v->succ(v)) < old.weight: \\
$\tab$$\tab$$\tab$change weight pred(v)->succ(v) to equal weight(pred(v)->v->succ(v))\\
$\tab$delete \textit{v}
\section*{Problem 1 - b}

\section*{Problem 1 - c}

To calculate all pairs' shortest paths from the above two algorithms, only simple modifications are required. For each v in V, run the algorithm in problem 1a, replacing "a random vertex \textit{v}" with "\textit{v}". After deleting a vertex \textit{v}, record it in an array. This step runs in O($V^3$) time since it runs an  O($V^2$) algorithm for each vertex. \\

Then, we have a list of vertices in the order we've deleted them. In LIFO order, perform algorithm b on each vertex replaced. at each step, algorithm 1b calculates the shortest path between the new vertex inserted and all existing vertices. This step runs in O($V^3$) time, since it runs an  O($V^2$) algorithm for each vertex. \\

Thus, the run-time for the total algorithm is  O($2V^3$), which simplifies to O($V^3$) \\
\end{document}